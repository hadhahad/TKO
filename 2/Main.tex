\documentclass[11pt,american,czech]{article}
\usepackage[T1]{fontenc}
\usepackage[utf8]{inputenc}
\usepackage[a4paper]{geometry}
\geometry{verbose,tmargin=2cm,bmargin=1cm,lmargin=1.5cm,rmargin=2cm,headheight=0.8cm,headsep=1cm,footskip=0.5cm}
\setcounter{secnumdepth}{3}
\usepackage{url}
\usepackage{amsmath}
\usepackage{amsthm}
\usepackage{amssymb}
\usepackage{graphicx}
\usepackage{setspace}

\usepackage{threeparttable}
\usepackage{array}

%% Use Times New Roman font for text and Belleek font for math
%% Please make sure that the 'esint' package is turned off in the
%% 'Math options' page.
\usepackage[varg]{txfonts}


%% Indent even the first paragraph in each section
\usepackage{indentfirst}

% completely avoid orphans (first lines of a new paragraph on the bottom of a page)
\clubpenalty=9500

% completely avoid widows (last lines of paragraph on a new page)
\widowpenalty=9500

% disable hyphenation of acronyms
\hyphenation{CDFA HARDI HiPPIES IKEM InterTrack MEGIDDO MIMD MPFA DICOM ASCLEPIOS MedInria}

%%---------------------------------------------------------------------

%% Print out all vectors in bold type instead of printing an arrow above them
\renewcommand{\vec}[1]{\boldsymbol{#1}}

% Replace standard \cite by the parenthetical variant \citep
%\renewcommand{\cite}{\citep}

\makeatother
\pagestyle{empty} %turns off page numbering

\usepackage{babel}
\begin{document}

\def\documentdate{\today}

%{\Large Domácí úkol č.2 z předmětu Teorie kódování. }  \par

\vspace{1cm}

\def\arraystretch{1.5}
%\setlength\tabcolsep{2.5mm}
\begin{table}[ht!]
\begin{threeparttable}
\caption{Sčítání v tělese $\Bbb{Z}_{\small 2}/_{x^3+x^2+1}$}
\begin{tabular}{|c||c|c|c|c|c|c|c|c|}
\hline
$\oplus$ & $x^2$ & $x^2+x$ & $x^2+x+1$ & $x^2+1$ & $x+1$ & $x$ & 1 & $0$ \\ \hline \hline
$x^2$ & $0$ & $x$ & $x+1$ & $1$ & $x^2+x+1$ & $x^2+x$ & $x^2+1$ & $x^2$ \\ \hline
$x^2+x$ & $x$ & $0$ & $1$ & $x+1$ & $x^2+1$ & $x^2$ & $x^2+x+1$ & $x^2+x$ \\ \hline
$x^2+x+1$ & $x+1$ & $1$ & $0$ & $x$ & $x^2$ & $x^2+1$ & $x^2+x$ & $x^2+x+1$ \\ \hline
$x^2+1$ & $1$ & $x+1$ & $x$ & $0$ & $x^2+x$ & $x^2+x+1$ & $x^2$ & $x^2+1$ \\ \hline
$x+1$ & $x^2+x+1$ & $x^2+1$ & $x^2$ & $x^2+x$ & $0$ & $1$ & $x$ & $x+1$ \\ \hline
$x$ & $x^2+x$ & $x^2$ & $x^2+1$ & $x^2+x+1$ & $1$ & $0$ & $x+1$ & $x$ \\ \hline
$1$ & $x^2+1$ & $x^2+x+1$ & $x^2+x$ & $x^2$ & $x$ & $x+1$ & $0$ & $1$ \\ \hline
$0$ & $x^2$ & $x^2+x$ & $x^2+x+1$ & $x^2+1$ & $x+1$ & $x$ & $1$ & $0$ \\ \hline
\end{tabular} 
\end{threeparttable}
\end{table}

\vspace{1cm}

%\setlength\tabcolsep{2.5mm}


%\newcolumntype{C}[1]{>{\centering\let\newline\\\arraybackslash\hspace{0pt}}m{#1}}

\newcolumntype{P}[1]{>{\centering\arraybackslash}p{#1}}

\begin{table}[h]
\begin{threeparttable}
\caption{Násobení v tělese $\Bbb{Z}_{\small 2}/_{x^3+x^2+1}$}
\begin{tabular}{|c||c|c|c|c|c|c|c| P{15mm} |}
\hline
$\otimes$ & $x^2$ & $x^2+x$ & $x^2+x+1$ & $x^2+1$ & $x+1$ & $x$ & $1$ & $0$ \\ \hline \hline
$x^2$ & $x^2+x+1$ & $x$ & $x^2+x$ & $x+1$ & $1$ & $x^2+1$ & $x^2$ & $0$ \\ \hline
$x^2+x$ & $x$ & $x+1$ & $x^2+1$ & $x^2$ & $x^2+x+1$ & $1$ & $x^2+x$ & $0$ \\ \hline
$x^2+x+1$ & $x^2+x$ & $x^2+1$ & $x$ & $1$ & $x^2$ & $x+1$ & $x^2+x+1$ & $0$ \\ \hline
$x^2+1$ & $x+1$ & $x^2$ & $1$ & $x^2+x$ & $x$ & $x^2+x+1$ & $x^2+1$ & $0$ \\ \hline
$x+1$ & $1$ & $x^2+x+1$ & $x^2$ & $x$ & $x^2+1$ & $x^2+x$ & $x+1$ & $0$ \\ \hline
$x$ & $x^2+1$ & $1$ & $x+1$ & $x^2+x+1$ & $x^2+x$ & $x^2$ & $x$ & $0$ \\ \hline
$1$ & $x^2$ & $x^2+x$ & $x^2+x+1$ & $x^2+1$ & $x+1$ & $x$ & $1$ & $0$ \\ \hline
$0$ & $0$ & $0$ & $0$ & $0$ & $0$ & $0$ & $0$ & $0$ \\ \hline
\end{tabular} 
%\begin{tablenotes}
%      \scriptsize
%      \item $^*$Operace je provedena pomocí ireducebilního polynomu $f(x)=x^3+x^2+1$.
%\end{tablenotes}
\end{threeparttable} 
\end{table}


\vfill
\begin{flushright}
Vladislav Belov
\end{flushright}
\pagebreak

\end{document}
